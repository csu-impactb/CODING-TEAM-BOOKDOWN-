% Options for packages loaded elsewhere
\PassOptionsToPackage{unicode}{hyperref}
\PassOptionsToPackage{hyphens}{url}
%
\documentclass[
]{book}
<<<<<<< HEAD
\title{ImpactTB/BAA: Standard Operating Procedures for Data Analysis}
\author{Colorado State University Coding Team}
\date{2022-06-10}

\usepackage{amsmath,amssymb}
=======
>>>>>>> 5397f63180927c0dc88558eb4437e03608b4249e
\usepackage{lmodern}
\usepackage{amsmath}
\usepackage{ifxetex,ifluatex}
\ifnum 0\ifxetex 1\fi\ifluatex 1\fi=0 % if pdftex
  \usepackage[T1]{fontenc}
  \usepackage[utf8]{inputenc}
  \usepackage{textcomp} % provide euro and other symbols
  \usepackage{amssymb}
\else % if luatex or xetex
  \usepackage{unicode-math}
  \defaultfontfeatures{Scale=MatchLowercase}
  \defaultfontfeatures[\rmfamily]{Ligatures=TeX,Scale=1}
\fi
% Use upquote if available, for straight quotes in verbatim environments
\IfFileExists{upquote.sty}{\usepackage{upquote}}{}
\IfFileExists{microtype.sty}{% use microtype if available
  \usepackage[]{microtype}
  \UseMicrotypeSet[protrusion]{basicmath} % disable protrusion for tt fonts
}{}
\makeatletter
\@ifundefined{KOMAClassName}{% if non-KOMA class
  \IfFileExists{parskip.sty}{%
    \usepackage{parskip}
  }{% else
    \setlength{\parindent}{0pt}
    \setlength{\parskip}{6pt plus 2pt minus 1pt}}
}{% if KOMA class
  \KOMAoptions{parskip=half}}
\makeatother
\usepackage{xcolor}
\IfFileExists{xurl.sty}{\usepackage{xurl}}{} % add URL line breaks if available
\IfFileExists{bookmark.sty}{\usepackage{bookmark}}{\usepackage{hyperref}}
\hypersetup{
  pdftitle={ImpactTB/BAA: Standard Operating Procedures for Data Analysis},
  pdfauthor={Colorado State University Coding Team},
  hidelinks,
  pdfcreator={LaTeX via pandoc}}
\urlstyle{same} % disable monospaced font for URLs
\usepackage{color}
\usepackage{fancyvrb}
\newcommand{\VerbBar}{|}
\newcommand{\VERB}{\Verb[commandchars=\\\{\}]}
\DefineVerbatimEnvironment{Highlighting}{Verbatim}{commandchars=\\\{\}}
% Add ',fontsize=\small' for more characters per line
\usepackage{framed}
\definecolor{shadecolor}{RGB}{248,248,248}
\newenvironment{Shaded}{\begin{snugshade}}{\end{snugshade}}
\newcommand{\AlertTok}[1]{\textcolor[rgb]{0.94,0.16,0.16}{#1}}
\newcommand{\AnnotationTok}[1]{\textcolor[rgb]{0.56,0.35,0.01}{\textbf{\textit{#1}}}}
\newcommand{\AttributeTok}[1]{\textcolor[rgb]{0.77,0.63,0.00}{#1}}
\newcommand{\BaseNTok}[1]{\textcolor[rgb]{0.00,0.00,0.81}{#1}}
\newcommand{\BuiltInTok}[1]{#1}
\newcommand{\CharTok}[1]{\textcolor[rgb]{0.31,0.60,0.02}{#1}}
\newcommand{\CommentTok}[1]{\textcolor[rgb]{0.56,0.35,0.01}{\textit{#1}}}
\newcommand{\CommentVarTok}[1]{\textcolor[rgb]{0.56,0.35,0.01}{\textbf{\textit{#1}}}}
\newcommand{\ConstantTok}[1]{\textcolor[rgb]{0.00,0.00,0.00}{#1}}
\newcommand{\ControlFlowTok}[1]{\textcolor[rgb]{0.13,0.29,0.53}{\textbf{#1}}}
\newcommand{\DataTypeTok}[1]{\textcolor[rgb]{0.13,0.29,0.53}{#1}}
\newcommand{\DecValTok}[1]{\textcolor[rgb]{0.00,0.00,0.81}{#1}}
\newcommand{\DocumentationTok}[1]{\textcolor[rgb]{0.56,0.35,0.01}{\textbf{\textit{#1}}}}
\newcommand{\ErrorTok}[1]{\textcolor[rgb]{0.64,0.00,0.00}{\textbf{#1}}}
\newcommand{\ExtensionTok}[1]{#1}
\newcommand{\FloatTok}[1]{\textcolor[rgb]{0.00,0.00,0.81}{#1}}
\newcommand{\FunctionTok}[1]{\textcolor[rgb]{0.00,0.00,0.00}{#1}}
\newcommand{\ImportTok}[1]{#1}
\newcommand{\InformationTok}[1]{\textcolor[rgb]{0.56,0.35,0.01}{\textbf{\textit{#1}}}}
\newcommand{\KeywordTok}[1]{\textcolor[rgb]{0.13,0.29,0.53}{\textbf{#1}}}
\newcommand{\NormalTok}[1]{#1}
\newcommand{\OperatorTok}[1]{\textcolor[rgb]{0.81,0.36,0.00}{\textbf{#1}}}
\newcommand{\OtherTok}[1]{\textcolor[rgb]{0.56,0.35,0.01}{#1}}
\newcommand{\PreprocessorTok}[1]{\textcolor[rgb]{0.56,0.35,0.01}{\textit{#1}}}
\newcommand{\RegionMarkerTok}[1]{#1}
\newcommand{\SpecialCharTok}[1]{\textcolor[rgb]{0.00,0.00,0.00}{#1}}
\newcommand{\SpecialStringTok}[1]{\textcolor[rgb]{0.31,0.60,0.02}{#1}}
\newcommand{\StringTok}[1]{\textcolor[rgb]{0.31,0.60,0.02}{#1}}
\newcommand{\VariableTok}[1]{\textcolor[rgb]{0.00,0.00,0.00}{#1}}
\newcommand{\VerbatimStringTok}[1]{\textcolor[rgb]{0.31,0.60,0.02}{#1}}
\newcommand{\WarningTok}[1]{\textcolor[rgb]{0.56,0.35,0.01}{\textbf{\textit{#1}}}}
\usepackage{longtable,booktabs}
\usepackage{calc} % for calculating minipage widths
% Correct order of tables after \paragraph or \subparagraph
\usepackage{etoolbox}
\makeatletter
\patchcmd\longtable{\par}{\if@noskipsec\mbox{}\fi\par}{}{}
\makeatother
% Allow footnotes in longtable head/foot
\IfFileExists{footnotehyper.sty}{\usepackage{footnotehyper}}{\usepackage{footnote}}
\makesavenoteenv{longtable}
\usepackage{graphicx}
\makeatletter
\def\maxwidth{\ifdim\Gin@nat@width>\linewidth\linewidth\else\Gin@nat@width\fi}
\def\maxheight{\ifdim\Gin@nat@height>\textheight\textheight\else\Gin@nat@height\fi}
\makeatother
% Scale images if necessary, so that they will not overflow the page
% margins by default, and it is still possible to overwrite the defaults
% using explicit options in \includegraphics[width, height, ...]{}
\setkeys{Gin}{width=\maxwidth,height=\maxheight,keepaspectratio}
% Set default figure placement to htbp
\makeatletter
\def\fps@figure{htbp}
\makeatother
\setlength{\emergencystretch}{3em} % prevent overfull lines
\providecommand{\tightlist}{%
  \setlength{\itemsep}{0pt}\setlength{\parskip}{0pt}}
\setcounter{secnumdepth}{5}
\usepackage{booktabs}
\usepackage{amsthm}
\makeatletter
\def\thm@space@setup{%
  \thm@preskip=8pt plus 2pt minus 4pt
  \thm@postskip=\thm@preskip
}
\makeatother
\ifluatex
  \usepackage{selnolig}  % disable illegal ligatures
\fi
\usepackage[]{natbib}
\bibliographystyle{apalike}

\title{ImpactTB/BAA: Standard Operating Procedures for Data Analysis}
\author{Colorado State University Coding Team}
\date{2022-06-09}

\begin{document}
\maketitle

{
\setcounter{tocdepth}{1}
\tableofcontents
}
\hypertarget{overview}{%
\chapter{Overview}\label{overview}}

Here, we have built a comprehensive guide to wet lab data collection, sample processing, and computational tool creation for robust and efficient data analysis and dissemination.

\hypertarget{introduction}{%
\chapter{Introduction}\label{introduction}}

\hypertarget{about-the-project-immune-mechanisms-of-protection-against-mycobacterium-tuberculosis-impac-tb}{%
\section{About the project: Immune Mechanisms of Protection against Mycobacterium tuberculosis (IMPAc-TB)}\label{about-the-project-immune-mechanisms-of-protection-against-mycobacterium-tuberculosis-impac-tb}}

The objective of the IMPAc-TB program is to get a thorough understanding of the immune responses necessary to avoid initial infection with \emph{Mycobacterium tuberculosis (Mtb)}, formation of latent infection, and progression to active TB illness. To achieve these goals, the National Institute of Allergy and Infectious Diseases awarded substantial funding and established multidisciplinary research teams that will analyze immune responses against \emph{Mtb} in animal models (mice, guinea pigs, and non-human primates) and humans, as well as immune responses elicited by promising vaccine candidates. The contract awards establish and give up to seven years of assistance for IMPAc-TB Centers to explain the immune responses required for \emph{Mtb} infection protection.

The seven centers that are part of the study are (in alphabetical order):

\begin{enumerate}
\def\labelenumi{\arabic{enumi}.}
\tightlist
\item
  Colorado State University
\item
  Harvard T.H. Chan School of Public Health
\item
  Seattle Children Hospital
\item
\end{enumerate}

<<<<<<< HEAD
Colorado State University Team and role of each member:
Dr.~Marcela Henao-Tamayo: Principal Investigator
Dr.~Brendan Podell: Principal Investigator
Dr.~Andres Obregon-Henao: Research Scientist-III
Dr.~Taru S. Dutt: Research Scientist-I
=======
Colorado State University Team:
>>>>>>> 5397f63180927c0dc88558eb4437e03608b4249e

\hypertarget{initial-mouse-characteristics}{%
\chapter{Initial mouse characteristics}\label{initial-mouse-characteristics}}

Here is a review of existing methods.

\hypertarget{mouse-weights}{%
\chapter{Mouse weights}\label{mouse-weights}}

Mice are weighed in grams weekly and recorded in an excel worksheet. Column titles are as follows: who\_collected date\_collected sex dob notch\_id mouse\_number weight unit cage\_number group notes

Groups included are: bcg, saline, bcg+id93, saline+id93, saline+noMtb

The notes column contains information regarding clinical observations.

\begin{Shaded}
\begin{Highlighting}[]
\FunctionTok{library}\NormalTok{(readxl)}
\FunctionTok{library}\NormalTok{(tidyverse)}
\end{Highlighting}
\end{Shaded}

\begin{verbatim}
## -- Attaching packages --------------------------------------- tidyverse 1.3.1 --
\end{verbatim}

\begin{verbatim}
## v ggplot2 3.3.5     v purrr   0.3.4
## v tibble  3.1.6     v dplyr   1.0.7
## v tidyr   1.1.4     v stringr 1.4.0
## v readr   2.1.1     v forcats 0.5.1
\end{verbatim}

\begin{verbatim}
## -- Conflicts ------------------------------------------ tidyverse_conflicts() --
## x dplyr::filter() masks stats::filter()
## x dplyr::lag()    masks stats::lag()
\end{verbatim}

\hypertarget{read-in-data}{%
\section{Read in data}\label{read-in-data}}

\begin{Shaded}
\begin{Highlighting}[]
\NormalTok{weight\_data }\OtherTok{\textless{}{-}} \FunctionTok{read\_xlsx}\NormalTok{(}\StringTok{"DATA/body\_weights.xlsx"}\NormalTok{)}
\end{Highlighting}
\end{Shaded}

\hypertarget{clean-data}{%
\section{Clean data}\label{clean-data}}

\begin{Shaded}
\begin{Highlighting}[]
\NormalTok{weight\_data }\SpecialCharTok{\%\textgreater{}\%}
  \FunctionTok{select}\NormalTok{(sex, mouse\_number, weight, cage\_number, group)}
\end{Highlighting}
\end{Shaded}

\begin{verbatim}
## # A tibble: 0 x 5
## # ... with 5 variables: sex <lgl>, mouse_number <lgl>, weight <lgl>,
## #   cage_number <lgl>, group <lgl>
\end{verbatim}

\hypertarget{summary-statistics}{%
\section{Summary statistics}\label{summary-statistics}}

\hypertarget{graph}{%
\section{Graph}\label{graph}}

\hypertarget{colony-forming-units-to-determine-bacterial-counts}{%
\chapter{Colony forming units to determine bacterial counts}\label{colony-forming-units-to-determine-bacterial-counts}}

\hypertarget{data-description}{%
\section{Data description}\label{data-description}}

The data are collected in a spreadsheet with multiple sheets. The first sheet
(named ``{[}x{]}'') is used to record some metadata for the experiment, while the
following sheets are used to record CFUs counts from the plates used for samples
from each organ, with one sheet per organ. For example, if you plated data
from both the lung and spleen, there would be three sheets in the file: one
with the metadata, one with the plate counts for the lung, and one with the
plate counts for the spleen.

The metadata sheet is used to record information about the overall process of
plating the data. Values from this sheet will be used in calculating the bacterial
load in the original sample based on the CFU counts. This spreadsheet includes
the following columns:

\begin{itemize}
\tightlist
\item
  \texttt{organ}: Include one row for each organ that was plated in the experiment.
  You should name the organ all in lowercase (e.g., ``lung'', ``spleen''). You
  should use the same name to also name the sheet that records data for that organ
  for example, if you have rows in the metadata sheet for ``lung'' and ``spleen'',
  then you should have two other sheets in the file, one sheet named ``lung'' and
  one named ``spleen'', which you'll use to store the plate counts for each of those
  organs.
\item
  \texttt{prop\_resuspended}: In this column, give the proportion of that organ that
  was plated. For example, if you plated half the lung, then in the ``lung'' row
  of this spread sheet, you should put 0.5 in the \texttt{prop\_resuspended} column.
\item
  \texttt{total\_resuspended\_uL}: This column contains an original volume of tissue homogenate. For example, raw lung tissue is homogenized in 500 uL of PBS in a tube containing metal beads.
\item
  \texttt{og\_aliquot\_uL}: 100 uL of th total\_resuspended slurry would be considered an original aliquot and is used to peform serial dilutions.
\item
  \texttt{dilution\_factor}: Amount of the original stock solution that is present in the total solution, after dilution(s)
\item
  \texttt{plated\_uL}: Amount of suspension + diluent plated on section of solid agar
\end{itemize}

\hypertarget{read-in-data-1}{%
\section{Read in data}\label{read-in-data-1}}

\begin{Shaded}
\begin{Highlighting}[]
\FunctionTok{library}\NormalTok{(readxl)}
\FunctionTok{library}\NormalTok{(dplyr)}
\FunctionTok{library}\NormalTok{(purrr)}
\FunctionTok{library}\NormalTok{(tidyr)}
\FunctionTok{library}\NormalTok{(stringr)}

\CommentTok{\#Replace w/ path to CFU sheet}
\NormalTok{path }\OtherTok{\textless{}{-}} \FunctionTok{c}\NormalTok{(}\StringTok{"DATA/Copy of baa\_cfu\_sheet.xlsx"}\NormalTok{)}

\NormalTok{sheet\_names }\OtherTok{\textless{}{-}} \FunctionTok{excel\_sheets}\NormalTok{(path)}
\NormalTok{sheet\_names }\OtherTok{\textless{}{-}}\NormalTok{ sheet\_names[}\SpecialCharTok{!}\NormalTok{sheet\_names }\SpecialCharTok{\%in\%} \FunctionTok{c}\NormalTok{(}\StringTok{"metadata"}\NormalTok{)]}

\NormalTok{merged\_data }\OtherTok{\textless{}{-}} \FunctionTok{list}\NormalTok{()}

\ControlFlowTok{for}\NormalTok{(i }\ControlFlowTok{in} \DecValTok{1}\SpecialCharTok{:}\FunctionTok{length}\NormalTok{(sheet\_names))\{}
  
\NormalTok{  data }\OtherTok{\textless{}{-}} \FunctionTok{read\_excel}\NormalTok{(path, }\AttributeTok{sheet =}\NormalTok{ sheet\_names[i]) }\SpecialCharTok{\%\textgreater{}\%} 
    \FunctionTok{mutate}\NormalTok{(}\AttributeTok{organ =} \FunctionTok{paste0}\NormalTok{(sheet\_names[i]))}
  
\NormalTok{  data }\OtherTok{\textless{}{-}}\NormalTok{ data }\SpecialCharTok{\%\textgreater{}\%} 
    \CommentTok{\#mutate(missing\_col = NA) \%\textgreater{}\% }
    \FunctionTok{mutate\_if}\NormalTok{(is.double, as.numeric) }\SpecialCharTok{\%\textgreater{}\%} 
    \FunctionTok{mutate\_if}\NormalTok{(is.numeric, as.character) }\SpecialCharTok{\%\textgreater{}\%} 
    \FunctionTok{pivot\_longer}\NormalTok{(}\FunctionTok{starts\_with}\NormalTok{(}\StringTok{"dil\_"}\NormalTok{), }\AttributeTok{names\_to =} \StringTok{"dilution"}\NormalTok{,}
                 \AttributeTok{values\_to =} \StringTok{"CFUs"}\NormalTok{) }\SpecialCharTok{\%\textgreater{}\%} 
    \FunctionTok{mutate}\NormalTok{(}\AttributeTok{dilution =} \FunctionTok{str\_extract}\NormalTok{(dilution, }\StringTok{"[0{-}9]+"}\NormalTok{),}
           \AttributeTok{dilution =} \FunctionTok{as.numeric}\NormalTok{(dilution))}
    
  
\NormalTok{  merged\_data[[i]] }\OtherTok{\textless{}{-}}\NormalTok{ data}
  
  
\NormalTok{\}}
  
\NormalTok{all\_data }\OtherTok{\textless{}{-}} \FunctionTok{bind\_rows}\NormalTok{(merged\_data, }\AttributeTok{.id =} \StringTok{"column\_label"}\NormalTok{) }\SpecialCharTok{\%\textgreater{}\%} 
    \FunctionTok{select}\NormalTok{(}\SpecialCharTok{{-}}\NormalTok{column\_label)}
\end{Highlighting}
\end{Shaded}

\hypertarget{exploratory-analysis-and-quality-checks}{%
\section{Exploratory analysis and quality checks}\label{exploratory-analysis-and-quality-checks}}

\hypertarget{exploratory-analysis}{%
\section{Exploratory analysis}\label{exploratory-analysis}}

\textbf{Dimensions of input data:}

Based on the input data, data were collected for the following organ or
organs:

The following number of mice were included for each:

The following number of replicates were recorded at each count date for
each experimental group:

The following number of dilutions and dilution level were recorded for
each organ:

\textbf{People who plated and collected the data. Date or dates of counting:}

Based on the input data, the plates included in these data were counted by
the following person or persons:
Based on the input data, the plates included in these data were counted on
the following date or dates:

\begin{Shaded}
\begin{Highlighting}[]
\NormalTok{all\_data }\SpecialCharTok{\%\textgreater{}\%}
  \FunctionTok{select}\NormalTok{(organ, who\_plated, who\_counted, count\_date) }\SpecialCharTok{\%\textgreater{}\%}
  \FunctionTok{distinct}\NormalTok{()}
\end{Highlighting}
\end{Shaded}

\begin{verbatim}
## # A tibble: 3 x 4
##   organ  who_plated who_counted count_date            
##   <chr>  <chr>      <chr>       <chr>                 
## 1 lung   BK         BK          "\"February 21 2022\""
## 2 lung   BK         BK          "\"April 18 2022\""   
## 3 spleen JR         JR          "\"April 25 2022\""
\end{verbatim}

\textbf{Distribution of CFUs at each dilution:}

WE NEED TO ADD SAMPLE CFU PLOTS

Here's a plot that shows how many plates were too numerous to count at each
dilution level:

Here is a plot that shows how the CFU counts were distributed by dilution
level in the data:

\hypertarget{identify-a-good-dilution-for-each-sample}{%
\section{Identify a good dilution for each sample}\label{identify-a-good-dilution-for-each-sample}}

\begin{Shaded}
\begin{Highlighting}[]
\CommentTok{\# Make all\_data into tidy data and filter for CFUs between 10{-}75}
  
\NormalTok{tidy\_cfu\_data }\OtherTok{\textless{}{-}}\NormalTok{ all\_data }\SpecialCharTok{\%\textgreater{}\%}
  \FunctionTok{mutate}\NormalTok{(}\AttributeTok{dilution =} \FunctionTok{str\_extract}\NormalTok{(dilution, }\StringTok{"[0{-}9]+"}\NormalTok{),}
         \AttributeTok{dilution =} \FunctionTok{as.numeric}\NormalTok{(dilution)) }\SpecialCharTok{\%\textgreater{}\%}
  \FunctionTok{filter}\NormalTok{(CFUs }\SpecialCharTok{\textgreater{}=} \DecValTok{10} \SpecialCharTok{\&}\NormalTok{ CFUs }\SpecialCharTok{\textless{}=} \DecValTok{75}\NormalTok{) }\SpecialCharTok{\%\textgreater{}\%}
  \FunctionTok{mutate}\NormalTok{(}\AttributeTok{CFUs =} \FunctionTok{as.numeric}\NormalTok{(CFUs))}
\end{Highlighting}
\end{Shaded}

\hypertarget{calculate-cfus-from-best-dilutionestimate-bacterial-load-for-each-sample-based-on-good-dilution}{%
\section{Calculate CFUs from best dilution/Estimate bacterial load for each sample based on good dilution}\label{calculate-cfus-from-best-dilutionestimate-bacterial-load-for-each-sample-based-on-good-dilution}}

\begin{Shaded}
\begin{Highlighting}[]
\CommentTok{\# Calculating CFU/ml for every qualifying replicate between 10{-}75 CFUs. Column binding by organ name to the metadata sheet via inner\_join().}
\NormalTok{meta }\OtherTok{\textless{}{-}} \FunctionTok{read\_excel}\NormalTok{(path, }\AttributeTok{sheet =} \StringTok{"metadata"}\NormalTok{)}

\NormalTok{tidy\_cfu\_meta\_joined }\OtherTok{\textless{}{-}} \FunctionTok{inner\_join}\NormalTok{(tidy\_cfu\_data, meta) }\SpecialCharTok{\%\textgreater{}\%}
  \FunctionTok{group\_by}\NormalTok{(groups) }\SpecialCharTok{\%\textgreater{}\%} 
  \FunctionTok{mutate}\NormalTok{(}\AttributeTok{CFUs\_per\_ml =}\NormalTok{ (CFUs }\SpecialCharTok{*}\NormalTok{ (dilution\_factor}\SpecialCharTok{\^{}}\DecValTok{2}\NormalTok{) }\SpecialCharTok{*}\NormalTok{ (total\_resuspension\_mL}\SpecialCharTok{/}\NormalTok{volume\_plated\_ul) }\SpecialCharTok{*} \DecValTok{10}\NormalTok{)) }\SpecialCharTok{\%\textgreater{}\%}
  \FunctionTok{select}\NormalTok{(organ, count\_date, who\_plated, who\_counted, groups,  mouse, dilution,  CFUs, CFUs\_per\_ml) }\SpecialCharTok{\%\textgreater{}\%}
  \FunctionTok{ungroup}\NormalTok{()}
\end{Highlighting}
\end{Shaded}

\begin{verbatim}
## Joining, by = "organ"
\end{verbatim}

\begin{Shaded}
\begin{Highlighting}[]
\NormalTok{tidy\_cfu\_meta\_joined}
\end{Highlighting}
\end{Shaded}

\begin{verbatim}
## # A tibble: 146 x 9
##    organ count_date           who_plated who_counted groups mouse dilution  CFUs
##    <chr> <chr>                <chr>      <chr>       <chr>  <chr>    <dbl> <dbl>
##  1 lung  "\"February 21 2022~ BK         BK          group~ A            3    53
##  2 lung  "\"February 21 2022~ BK         BK          group~ A            5     4
##  3 lung  "\"February 21 2022~ BK         BK          group~ A            6     2
##  4 lung  "\"February 21 2022~ BK         BK          group~ B            3   119
##  5 lung  "\"February 21 2022~ BK         BK          group~ B            4    48
##  6 lung  "\"February 21 2022~ BK         BK          group~ B            5    18
##  7 lung  "\"February 21 2022~ BK         BK          group~ C            3   120
##  8 lung  "\"February 21 2022~ BK         BK          group~ C            4    32
##  9 lung  "\"February 21 2022~ BK         BK          group~ D            3    53
## 10 lung  "\"February 21 2022~ BK         BK          group~ D            4    31
## # ... with 136 more rows, and 1 more variable: CFUs_per_ml <dbl>
\end{verbatim}

\hypertarget{create-initial-report-information-for-these-data}{%
\section{Create initial report information for these data}\label{create-initial-report-information-for-these-data}}

\hypertarget{sample-anova}{%
\section{Sample ANOVA}\label{sample-anova}}

\begin{Shaded}
\begin{Highlighting}[]
\NormalTok{cfu\_stats }\OtherTok{\textless{}{-}}\NormalTok{ tidy\_cfu\_meta\_joined }\SpecialCharTok{\%\textgreater{}\%} 
  \FunctionTok{group\_by}\NormalTok{(organ) }\SpecialCharTok{\%\textgreater{}\%}
  \FunctionTok{nest}\NormalTok{() }\SpecialCharTok{\%\textgreater{}\%}
  \FunctionTok{mutate}\NormalTok{(}\AttributeTok{aov\_result =} \FunctionTok{map}\NormalTok{(data, }\SpecialCharTok{\textasciitilde{}}\FunctionTok{aov}\NormalTok{(CFUs\_per\_ml }\SpecialCharTok{\textasciitilde{}}\NormalTok{ groups, }\AttributeTok{data =}\NormalTok{ .x)),}
         \AttributeTok{tukey\_result =} \FunctionTok{map}\NormalTok{(aov\_result, TukeyHSD),}
         \AttributeTok{tidy\_tukey =} \FunctionTok{map}\NormalTok{(tukey\_result, broom}\SpecialCharTok{::}\NormalTok{tidy)) }\SpecialCharTok{\%\textgreater{}\%}
  \FunctionTok{unnest}\NormalTok{(tidy\_tukey, }\AttributeTok{.drop =} \ConstantTok{TRUE}\NormalTok{) }\SpecialCharTok{\%\textgreater{}\%}
  \FunctionTok{separate}\NormalTok{(contrast, }\AttributeTok{into =} \FunctionTok{c}\NormalTok{(}\StringTok{"contrast1"}\NormalTok{, }\StringTok{"contrast2"}\NormalTok{), }\AttributeTok{sep =} \StringTok{"{-}"}\NormalTok{) }\SpecialCharTok{\%\textgreater{}\%}
  \FunctionTok{select}\NormalTok{(}\SpecialCharTok{{-}}\NormalTok{data, }\SpecialCharTok{{-}}\NormalTok{aov\_result, }\SpecialCharTok{{-}}\NormalTok{tukey\_result, }\SpecialCharTok{{-}}\NormalTok{term, }\SpecialCharTok{{-}}\NormalTok{null.value)}\CommentTok{\# \%\textgreater{}\%}
\end{Highlighting}
\end{Shaded}

\begin{verbatim}
## Warning: The `.drop` argument of `unnest()` is deprecated as of tidyr 1.0.0.
## All list-columns are now preserved.
## This warning is displayed once every 8 hours.
## Call `lifecycle::last_lifecycle_warnings()` to see where this warning was generated.
\end{verbatim}

\begin{Shaded}
\begin{Highlighting}[]
  \CommentTok{\# filter(adj.p.value \textless{}= 0.05)}

\NormalTok{cfu\_stats}
\end{Highlighting}
\end{Shaded}

\begin{verbatim}
## # A tibble: 9 x 7
## # Groups:   organ [2]
##   organ  contrast1 contrast2 estimate conf.low conf.high adj.p.value
##   <chr>  <chr>     <chr>        <dbl>    <dbl>     <dbl>       <dbl>
## 1 lung   group_2   group_1     -15.0     -39.4      9.34       0.377
## 2 lung   group_3   group_1     -13.1     -39.2     13.1        0.562
## 3 lung   group_4   group_1      -2.57    -27.1     22.0        0.993
## 4 lung   group_3   group_2       1.98    -22.7     26.7        0.997
## 5 lung   group_4   group_2      12.5     -10.5     35.5        0.491
## 6 lung   group_4   group_3      10.5     -14.4     35.4        0.689
## 7 spleen group_2   group_1     -21.5     -48.8      5.80       0.146
## 8 spleen group_3   group_1     -17.6     -45.9     10.7        0.294
## 9 spleen group_3   group_2       3.90    -23.4     31.2        0.935
\end{verbatim}

\hypertarget{save-processed-data-to-database}{%
\section{Save processed data to database}\label{save-processed-data-to-database}}

\hypertarget{example-one}{%
\section{Example one}\label{example-one}}

\hypertarget{example-two}{%
\section{Example two}\label{example-two}}

\hypertarget{elisa-words}{%
\chapter{ELISA Words}\label{elisa-words}}

We have finished a nice book.

  \bibliography{book.bib,packages.bib}

\end{document}
